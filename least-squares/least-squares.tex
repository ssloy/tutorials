\documentclass[A4,11pt]{article}

\usepackage[T2A]{fontenc}
\usepackage[utf8]{inputenc}
\usepackage[russian]{babel}


\usepackage{amsmath,amssymb,amsfonts}


\begin{document}
Кратко о наименьших квадратах

Полтора года назад я опубликовал статью «Математика на пальцах: методы наименьших квадратов»,
https://habr.com/post/277275/
которая получила весьма приличный отклик,
который, в том числе, заключался в том, что я предложил нарисовать сову.
Раз сова, значит, нужно объяснять ещё раз. А через неделю ровно на эту тему я начну читать несколько лекций студентам-геологам;
пользуюсь случаем, излагаю тут (адаптированные) основные тезисы в качестве черновика.
Я разобью текст на (наверное) три статьи:
- Ликбез по теорвер
- Наименьшие квадраты, простейший случай
- Нелинейные МНК
Основная идея


Я зайду к наименьшим квадратам чуть сбоку, через принцип максимума правдоподобности, а он требует минимального ориентирования в теории вероятностей.
Данный текст рассчитан на третий курс нашего факультета геологии, что означает, (с точки зрения задействованного матаппарата!) что заинтересованный старшеклассник 
при соответствующем усердии должен суметь в нём разобраться.

\section{Ликбез по теории вероятностей}
\subsection{Верите ли вы в теорию эволюции?}
Однажды мне задали вопрос, верю ли я в теорию эволюции. Прямо сейчас сделайте паузу, подумайте, как вы на него ответите.

Лично я (опешив) ответил, что нахожу её правдоподобной. Научная теория имеет мало общего с верой.
Если кратко, то теория лишь строит модель окружающего нас мира, нет необходимости в неё верить.
Состоятельная теория должна обладать, в первую очередь, предсказательной силой.
Например, если вы генетически модифицируете сельскохозяйственные культуры таким образом, что они сами будут производить пестициды,
то вполне логично, что будут появляться устойчивые к ним насекомые.
Однако существенно менее очевидно, что этот процесс может быть замедлен благодаря выращиванию обычных растений бок-о-бок с генномодифицированными.
Основываясь на теории эволюции, соответствующее моделирование сделало такой прогноз,

https://www.sciencedirect.com/science/article/pii/S0167880903002858?via%3Dihub

и он, похоже, оправдался.

https://www.nature.com/articles/nbt1382



Теории бывают самыми разными и 


Дальше, конечно, можно углубляться в философию



Фреквентисты,
Qu'est-ce qui se passe si on les respecte pas?

чемпионат мира: уругвай=a германия=b

\begin{alignat*}{2}
P(a) = .4 & \qquad & P(a\wedge b) = 0 \\
P(b) = .3 &        & P(a\vee b) = .8
\end{alignat*}
%$P(a) = .4  \qquad  P(a\wedge b) = 0 \qquad P(b) = .3 \qquad P(a\vee b) = .8$
\begin{tabular}{cccccc}
\multicolumn{2}{c}{Ставки Агента 2}  &  \multicolumn{4}{c}{Результат для Агента 1} \\
\hline
{\tiny Событие} & {\tiny Сумма ставки} & {\tiny $a\wedge b$} & {\tiny $a\wedge \neg b$} &  {\tiny $\neg a\wedge b$} &  {\tiny $\neg a\wedge\neg b$} \\
\hline
$a$             & 4-6 & -6 & -6 &  4 &  4 \\
$b$             & 3-7 & -7 &  3 & -7 &  3 \\
$\neg(a\vee b)$ & 2-8 &  2 &  2 &  2 & -8 \\
\hline
                &     &-11 & -1 & -1 & -1
\end{tabular}

Агент 1 теряет деньги при любом раскладе!

\subsection{Что изучает теорвер}

Il y a 3 axiomes de Kolmogorov :
\begin{itemize}
\item $0\leq P(a)\leq 1$
\item $P(true)=1$, $P(false) = 0$
\item $P(a\vee b) = P(a) + P(b) - P(a\wedge b)$
\end{itemize}

\subsection{Плотность вероятности}

\subsection{Условная вероятность}

\subsection{Правдоподобие}




\end{document}
